\begin{abstract}
Transitioning to High Assay Low Enriched Uranium-fueled reactors will
alter the material requirements of the current nuclear fuel cycle, in 
terms of the mass of enriched uranium and Separative Work Unit capacity.  
This work simulates multiple fuel cycle scenarios to compare how the type of 
the advanced reactor deployed and the energy growth demand affect the 
material requirements of the transition to High Assay Low Enriched 
Uranium-fueled reactors. Fuel 
cycle scenarios considered include a no-growth and a 1\% growth transition to 
either the Ultra Safe Nuclear Corporation Micro Modular Reactor\textsuperscript{TM} 
or the X-energy Xe-100 reactor
from the current fleet of U.S. Light Water Reactors. Materials of interest include 
the number of 
advanced reactors deployed, the mass of enriched uranium sent to the reactors, 
and the Separative Work Unit capacity required to enrich natural uranium for the reactors.
Deploying Micro Modular Reactor\textsuperscript{TM}s requires a higher 
average mass and Separative Work Unit capacity than deploying Xe-100 reactors, and a lower 
enriched uranium mass and a higher Separative Work Unity capacity than required to fuel 
Light Water Reactors before the transition. Fueling Xe-100 reactors requires less 
enriched uranium and Separative Work Unit capacity than fueling Light Water Reactors 
before the transition. 
\end{abstract}

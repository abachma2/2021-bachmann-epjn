\section{Methodology}
This work simulates multiple transition scenarios to advanced reactors
requiring \gls{HALEU} for fuel, then quantifies and compares the front-end 
material requirements of each scenario. Five different fuel cycle scenarios 
are modeled using \Cyclus \cite{huff_fundamental_2016};
an open-source, agent-based fuel cycle simulator. \Cyclus defines facilities, 
institutions, and regions as agents within a fuel cycle simulation and models 
material transactions between agents according to a dynamic resource exchange. 

The first scenario models 
only the \glspl{LWR} that are deployed in the United States, and provides 
a reference for comparison of the material requirements of the transitions. 
The next two scenarios model no-growth 
transitions to the \gls{USNC} \gls{MMR} or the X-Energy 
Xe-100 reactor. The last  
two scenarios model a 1\% annual growth transition to the \gls{USNC} 
\gls{MMR} or the X-Energy Xe-100 reactor. Table \ref{tab:simulations} 
summarizes each of the scenarios.

\begin{table}[ht]
        \centering
        \caption{Summary of the fuel cycle scenarios}
        \label{tab:simulations}
        \begin{tabular}{c c c}
                \hline
                Scenario Number & Reactors Present & Growth \\\hline
                1 & \glspl{LWR} & N/A \\
                2 & \glspl{LWR} and \gls{USNC} \gls{MMR} & None \\
                3 & \glspl{LWR} and X-energy Xe-100 reactor& None \\
                4 & \glspl{LWR} and \gls{USNC} \gls{MMR}& 1\% \\
                5 & \glspl{LWR} and X-energy Xe-100 reactor& 1\% \\\hline

        \end{tabular}
\end{table}

The \gls{IAEA} \gls{PRIS} database \cite{noauthor_power_1989} supplies
data about the currently deployed \glspl{LWR}. This database provides the 
grid connection date and power level of each reactor deployed, and the 
decommission 
date for any reactor closed before December 2020. Any reactor still in 
operation in December 2020 is assumed to operate for 60 years after its 
grid connection date. Only reactors with a power level above 400 MWe are included 
to ensure that no research or experimental reactors are used. 
The mass of fuel in the \gls{LWR} reactor cores, including the mass  
required for each refueling, was obtained from supplementary sources 
\cite{todreas_nuclear_2012,cacuci_handbook_2010}.
All \glspl{LWR} are assumed to have an 18 month cycle length. 

We used a variety of open-source documents to obtain data about the advanced reactors
\cite{mitchell_usnc_2020, hawari_development_2018, venneri_neutronic_2015, 
harlan_x-energy_2018, hussain_advances_2018}. 
This includes 
the power output, enrichment level, fuel form, reactor lifetime, fuel 
burnup, and cycle time, shown in Table \ref{tab:reactor_summary}. 

Knowing a reactor \gls{EOL} burnup, power output, and its cycle length
allows for calculating its initial load of fissile materials.
In the MMR case, considering a burnup of 42.7 MWd/kgU \cite{hawari_development_2018},
a power output of 40 MWth, and a cycle length of 2,042 EFPD \cite{venneri_neutronic_2015}
results in the batch mass shown in Table \label{tab:reactor_summary}.
The following simulations assume that this value remains constant for the
power output and cycle length specified in the same table.

The mass of uranium for the fresh \gls{TRISO} pebbles for the Xe-100 
reactor was found by calculating the total volume of UCO \gls{TRISO} particle
kernell in a fuel pebble the number of pebbles in the Xe-100 core, then 
multiplying by the density of UCO and the mass fraction of uranium. 
The isotopic compositions of the spent pebbles in the Xe-100 were found using a simple 
Serpent 2 cite %https://serpent.vtt.fi/mediawiki/index.php/Main_Page 
simulation of a single fuel pebble in a cube of helium, using a 
reflective boundary condition.  The Serpent 2 depletion model finds the burnup 
and isotopic compositions at six time steps, each corresponding to the first, 
second, third, fourth, fifth, and sixth six-month pass, respectively.  A 
complete list of isotopic compositions for each level of burnup, in atomic 
fraction, can be found at cite %https://zenodo.org/record/5501385#.YTzUv1tOk1g.

The Xe-100 reactor has a larger power output, requires a higher enrichment 
level, and has a longer lifetime than the \gls{MMR}. However, the 
\gls{MMR} has a longer cycle time than the Xe-100 since it does not require 
refueling once it is operational. Both advanced reactors require fuel 
comprised of \gls{TRISO} particles, but in different forms. Refueling 
of the Xe-100 reactor is modeled as a replacement of 1/7th of the core mass 
every six months.  

\begin{table}[ht]
        \caption{Advanced reactor design specifications}
        \label{tab:reactor_summary}
        \begin{tabular}{l c c c }
            \hline
            Design Criteria & \gls{USNC} \gls{MMR} \cite{mitchell_usnc_2020}& 
                X-Energy Xe-100 \cite{harlan_x-energy_2018,hussain_advances_2018} \\\hline
            Reactor type & Modular HTGR & Modular HTGR \\
            Power Output (MWe) & 10 & 75 \\
            Enrichment (\% $^{235}U$) & 13 & 15.5 \\
            Cycle Length (years) & 20 & online refueling\\
            Fuel form & \gls{TRISO} compacts & \gls{TRISO} pebbles\\
            Reactor Lifetime & 20 years & 60 years \\
            Mass of uranium per refueling (kg) & 1912.9 & 223.87 \\
            Burnup (MWd/kg U) & 42.7 & 163 \\
            \hline
        \end{tabular}
    \end{table}

Each of the simulations model reactor deployment and operation from 1965 to 
2090, with the transition to advanced reactors beginning 
in 2025 for the applicable scenarios. Therefore, in the no growth scenarios 
the power demand is held constant 
at the power produced in 2025. The energy demand of each transition scenario 
(Scenarios 2-5) is modeled by either a linear (for no growth) or 
exponential (for 1\% growth) equation defined by the \Cycamore GrowthRegion
archetype \cite{huff_fundamental_2016}, which determines if additional 
facilities are required to meet the specified demand. The \glspl{LWR} are deployed 
using the \Cycamore DeployManagerInst archetype, and the 
advanced reactors are deployed as needed to meet the prescribed power demand 
of the scenario using the \Cycamore ManagerInst archetype 
\cite{huff_fundamental_2016}. The \Cycamore DeployManagerInst deploys facilities 
according to a manually descriped schedule, and each facility deployed is 
recognized as part of the \Cycamore GrowthRegion and how each facility contributes 
to the specified capacity of the scenario. 

The scenarios model the fuel cycle from the uranium mine to the final 
disposal of the fuel in the HLW Sink.  Figure 
\ref{fig:fuel_cycle} shows the fuel cycle modeled in each simulation. Scenario 
1 only includes the facilities in blue. All facilities are used in Scenarios 
2-5, and the facility in red is the advanced reactor deployed in the 
scenario. Although the back end of the fuel 
cycle is modeled, quantifying any waste is considered outside the 
scope of this work. 

\begin{figure}[ht]
        \centering
        \begin{tikzpicture}[node distance=1.5cm]
            \node (mine) [facility] {Uranium Mine};
            \node (mill) [facility, below of=mine] {Uranium Mill};
            \node (conversion) [facility, below of=mill] {Conversion};
            \node (enrichment) [facility, below of=conversion]{Enrichment};
            \node (fabrication) [facility, below of=enrichment]{Fuel Fabrication};
            \node (reactor) [facility, below of=fabrication, xshift=-2cm]{LWR};
            \node (adv_reactor) [transition, below of=fabrication, xshift=2cm]{Advanced Reactor};
            \node (wetstorage) [facility, below of=reactor]{Wet Storage};
            \node (drystorage) [facility, below of=wetstorage]{Dry Storage};
            \node (sinkhlw) [facility, below of=drystorage, xshift=2cm]{HLW Sink};
            \node (sinkllw) [facility, right of=enrichment, xshift=2cm] {LLW Sink};
    
            \draw [arrow] (mine) -- node[anchor=east]{Natural U} (mill); 
            \draw [arrow] (mill) -- node[anchor=east]{U$_3$O$_8$}(conversion); 
            \draw [arrow] (conversion) -- node[anchor=east]{UF$_6$}(enrichment);
            \draw [arrow] (enrichment) -- node[anchor=east]{Enriched U}(fabrication);
            \draw [arrow] (enrichment) -- node[anchor=south]{Tails}(sinkllw);
            \draw [arrow] (fabrication) -- node[anchor=east]{Fresh UOX}(reactor);
            \draw [arrow] (fabrication) -- node[anchor=west]{TRSIO fuel}(adv_reactor);
            \draw [arrow] (reactor) -- node[anchor=east]{Spent UOX}(wetstorage);
            \draw [arrow] (wetstorage) -- node[anchor=east]{Cool Spent UOX}(drystorage);
            \draw [arrow] (drystorage) -- node[anchor=east]{Casked Spent UOX}(sinkhlw);
            \draw [arrow] (adv_reactor) -- node[anchor=west]{Spent TRISO Fuel}(sinkhlw);
    
            \end{tikzpicture}
        \caption{Fuel cycle facilities and material flow between facilities. Facilities in 
        red are deployed in the transition scenarios.}
        \label{fig:fuel_cycle}
    \end{figure}

The composition of the materials shown in Figure \ref{fig:fuel_cycle} 
are defined using recipes. Spent and fresh \gls{LWR} 
fuel recipes were obtained from \cite{yacout_visionverifiable_2006}. 
The spent \gls{LWR} fuel 
recipe assumes a 51 MWd/kg-U burnup. All other recipes capture the 
necessary isotopic ratios of uranium, but do not include other elements.
Neutronic or depletion simualtions were not modeled as part of this work.  
The enrichment feed recipe assumes natural uranium is used, and the tails 
assay is 0.2\%. 
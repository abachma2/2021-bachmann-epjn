\section{Conclusions}
This work simulated multiple fuel cycle scenarios to compare the 
material requirements of deploying different advanced reactors fueled 
by \gls{HALEU}. Scenarios include a no-growth and a 1\% growth 
transitions to either the \gls{USNC} \gls{MMR} or the X-energy Xe-100 
reactor from the current fleet of U.S. \glspl{LWR}. We used the current 
fleet of \glspl{LWR}, without a transition to an advanced reactor 
fleet for 
comparison. Each of these scenarios are compared for their material 
requirements, specifically the number of reactors deployed, the mass 
of enriched uranium sent to the reactors, and the \gls{SWU} capacity 
required to enrich natural uranium to produce the fuel needed for 
each scenario. 

More \glspl{MMR} are required than Xe-100 reactors to meet the same 
energy demand, because of the differences in their power output. 
The transition scenarios exhibit some gaps between  
energy production and demand, localized to the beginning of the 
transition
scenarios, and to the replacement of \glspl{MMR} as they are decommissioned. 
Xe-100 reactors are not decommissioned in the simulated time frame because 
their lifetime exceeds the time span simulated. Therefore, the 
energy produced and the material requirements for the replacement of 
Xe-100 reactors is not explored in this work. 

Transitioning to the \gls{MMR} requires 
a larger average mass and \gls{SWU} capacity than transitioning to the 
Xe-100 reactors. Transitioning to the \gls{MMR} requires a smaller average 
uranium mass but a greater \gls{SWU} capacity than fueling \glspl{LWR} prior 
to 2025 because the \gls{MMR} requires a higher enrichment level. 
Fueling \glspl{MMR} involves large, one-time shipments of fuel, while 
fueling the Xe-100 involves small, continuous shipments of fuel 
because of the different refueling schemes.

\section{Future Work}
One possible area of future work is extending the end date 
of the scenarios to 2125 to investigate how replacing deployed Xe-100 
reactors impacts the material requirements and energy produced in those 
transition scenarios. Another area of future work is to investigate the 
material requirements if the \gls{HALEU} demand is met by downblending 
\gls{HEU} or enriching uranium of non-natural enrichment, such as 
\gls{LEU} below 5\% $^{235}$U. Finally, an investigation into the 
optimization of an enrichment facility or the use of other methods to 
create \gls{HALEU} to meet the demand of each scenario will provide 
insight into the logistics of transitioning to \gls{HALEU}-fueled 
reactors. 

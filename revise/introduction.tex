\section{Introduction}

Most \glspl{LWR} operating in the U.S. are slated to retire
before 2050; meaning that if nuclear power is to continue providing a 
significant portion of energy in the U.S., new reactors will need to be built. 
New reactors are likely to be advanced reactor designs, many of 
which require \gls{HALEU} for fuel. \gls{HALEU} is uranium  
enriched between 5-20\% $^{235}$U, compared to the \gls{LEU} enriched to 
less than 5\% $^{235}$U that fuels current \glspl{LWR}. \gls{HALEU} fuel helps 
these advanced reactors achieve higher burnups and longer cycle times than current 
\glspl{LWR}. However, changes to the fuel enrichment change the material 
requirements of their fuel cycled, and may delay 
new reactor deployment.

The current supply chain for \gls{LEU} below 5\% $^{235}$U 
enriches \gls{NU} to the required level, and there is no commercial 
supply chain for \gls{HALEU}.  
The mass of uranium that can be mined and the \gls{SWU} capacity available 
to enrich it limits the mass of enriched uranium that can be 
produced. Understanding 
the transition to \gls{HALEU}-fueled reactors will inform the material 
requirements to meet electricity demands and 
the potential rate of reactor deployment.

In 2011 the \gls{DOE-NE} commissioned a Nuclear Fuel Cycle \gls{ES} 
\cite{wigeland_nuclear_2014} to compare fuel cycle options at equilibrium
to determine their advantages and disadvantages. The fuel cycle options were 
compared 
based on 9 high-level performance metrics, such as resource utilization and 
environmental impact. Each of the fuel cycle options included in the 
\gls{ES} fall into one of 40 different \glspl{EG}, defined by fuel 
characteristics such as the fuel type, neutron spectrum, and the inclusion 
of spent fuel recycling. 

One of the \glspl{EG}, \gls{EG}02 -- ``Once-through using enriched-U fuel to 
high burnup in thermal or fast critical reactors'' -- evaluated the metrics 
of a future fuel cycle with a \gls{HTGR} using uranium fuel enriched to 
15.5\% and achieving 120 GWd/t burnup. The \gls{ES} found that this fuel cycle
requires about 20,000 t of \gls{NU} and about 600 t of fuel at 15.5\% 
enrichment each year 
to produce 100 GWe-yr at equilibrium. Almost 19,000 t of \gls{NU} and 
about 2,000 t of uranium fuel at 4.21\%
enrichment are required in \gls{EG}01 -- ``Once-through using enriched-U 
fuel in thermal critical reactors'' -- which models the continued use of current 
\glspl{LWR}. This shows an increase in \gls{NU} and a decrease in enriched uranium 
mass are needed to deploy reactors using \gls{HALEU}.
A similar effect was observed by increasing the enrichment of fuel in a 
\gls{PWR} from 4\% to 7\% \cite{burns_reactor_2020}, and in a \gls{LWR}
\gls{SMR} of Russian origin \cite{hernandez_potential_2020}. 

Based on prior work, we expect increasing the enrichment level of uranium fuel 
to increase the material requirements at the front end of 
the fuel cycle. However, the exact change in requirements depends 
on the type of reactor, enrichment level, and transition scenario.
Modeling the transition to any future
reactor design can quantify the material requirements of the transition
and help us understand if the current supply chain is sufficient to meet them. 
Comparing the material requirements of each transition scenario will 
reveal any advantages or disadvantages of a given transition scenario. 

This work aims to quantify the material resource requirements at the front 
end of the 
transition to different types of \gls{HALEU}-fueled advanced reactors, 
assuming that \gls{NU} will be enriched to produce fuel for \glspl{LWR} and 
\gls{HALEU}-fueled reactors. 
Metrics of interest include the deployment schedule of \gls{HALEU}-fueled 
reactors, mass of enriched uranium, and the \gls{SWU} capacity required to 
enrich uranium for each scenario. 
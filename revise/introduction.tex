\section{Introduction}

Most of the \glspl{LWR} operating in the U.S. are slated to retire
before 2050; meaning that if nuclear power is to continue providing a 
significant portion of energy in the U.S., new reactors will need to be built. 
New reactors are likely to be an advanced reactor design, many of 
which require \gls{HALEU} to fuel them. \gls{HALEU} is uranium that 
is enriched between 5-20\% $^{235}$U, compared to the less than 5\% $^{235}$U 
that fuels current \glspl{LWR}. \gls{HALEU} fuel will help 
the reactors achieve higher burnups and longer cycle times than current 
\glspl{LWR}. However, changes to the fuel enrichment are 
expected 
to change the material requirements of the fuel cycle, and may delay 
new reactor design deployment.

The current supply chain for \gls{LEU} below 5\% enrichment is based on 
enriching \gls{NU} to the required level, and there is no commercial 
supply chain for \gls{HALEU}. There are
two primary methods for creating \gls{HALEU}: \gls{HEU} recovery and 
down-blending, 
and the enrichment of \gls{NU}. Downblending \gls{HEU} is 
limited by the physical inventory of \gls{HEU} available. 
Enriching \gls{NU} is limited by the amount of uranium that can be 
mined and the \gls{SWU} capacity to enrich the uranium. Understanding the 
material requirements of the transition to \gls{HALEU}-fueled 
reactors will provide insight into how each of these methods can be used 
to obtain the \gls{HALEU} fuel needed to meet electricity demands and 
how the production method affects the rate of reactor deployment.

In 2011 the \gls{DOE-NE} commissioned a Nuclear Fuel Cycle \gls{ES} 
\cite{wigeland_nuclear_2014} to compare fuel cycle options at equilibrium
to determine their advantages and disadvantages
based on 9 high-level performance metrics, such as resource utilization and 
environmental impact. Each of the fuel cycle options included in the 
\gls{ES} fall into one of 40 different \glspl{EG}, defined by fuel 
characteristics such as the type of fuel in the reactors (uranium, 
U/Th), neutron spectrum, and the inclusion of spent fuel recycling. 

One of the \glspl{EG}, \gls{EG}02 - ``Once-through using enriched-U fuel to 
high burnup in thermal or fast critical reactors'', looked at the metrics 
of a future fuel cycle with a \gls{HTGR} using uranium fuel enriched to 
15.5\% and achieving 120 GWd/t burnup. The \gls{ES} found that this fuel cycle
requires about 20,000 MT of \gls{NU} and about 600 MT of fuel each year 
to produce 100 GWe-yr of electricity at equilibrium. This is more than the 
almost 19,000 MT of \gls{NU} and about 2,000 MT of uranium fuel at 4.21\%
enrichment that would be required in \gls{EG}01 - ``Once-through using enriched-U 
fuel in thermal critical reactors'', which models the continued use of current 
\glspl{LWR}. This shows an increase in \gls{NU} and fuel mass is needed to 
deploy reactors using \gls{HALEU}.
A similar effect was observed by increasing the enrichment of fuel in a 
\gls{PWR} from 4\% to 7\% \cite{burns_reactor_2020}, and in a \gls{LWR}
\gls{SMR} of Russian origin \cite{hernandez_potential_2020}. 

Based on this prior work, increasing the enrichment level of uranium fuel 
is expected to  
increase the material requirements at the front end of 
the fuel cycle. However, the exact change in requirements depends 
on the type of reactor, enrichment level, and transition scenario.
Therefore, it is important to accurately model the transition to any future
reactor designs to quantify the material requirements and understand
if the current supply chain is sufficient to meet them. 

This work aims to quantify the material resource requirements at the front 
end of the 
transition to different types of \gls{HALEU}-fueled advanced reactors, 
assuming that \gls{NU} will be enriched to produce fuel for \glspl{LWR} and 
\gls{HALEU}-fueled reactors. 
Metrics of interest include the deployment schedule of \gls{HALEU}-fueled 
reactors, mass of enriched uranium, and the \gls{SWU} capacity required to 
enrich uranium for each scenario. The material requirements of each transition 
scenario will be compared to identify any advantages or disadvantages of 
a given transition scenario. 
\section{Conclusion}
This work simulated multiple fuel cycle scenarios to compare the 
material requirements of deploying different advanced reactors fueled 
by \gls{HALEU}. Scenarios include a no-growth and a 1\% growth 
transitions to either the \gls{USNC} \gls{MMR} or the X-energy Xe-100 
reactor from the current fleet of \glspl{LWR} in the US. The current 
fleet of \glspl{LWR}, without a transition to an advanced reactor, is 
also simulated. Each of these scenarios are compared for their material 
requirements, specifically the number of reactors deployed, the mass 
of enriched uranium sent to the reactors, and the \gls{SWU} capacity 
required to enrich natural uranium to produce the fuel needed for 
each scenario. 

The transition scenarios exhibit some gaps between  
energy production and demand, localized to the beginning of the 
transition
scenarios, and to the replacement of \glspl{MMR} as they are decommissioned. 
Xe-100 reactors are not decommissioned in the simulated time frame because 
their lifetime exceeds the time span simulated. Therefore, the 
energy produced and the material requirements for the replacement of 
Xe-100 reactors is not explored in this work. 

This work shows that transitioning to the \gls{MMR} requires 
a higher average mass and \gls{SWU} capacity than transitioning to the 
Xe-100 reactors, but a lower average than what is required by the current 
\glspl{LWR}. Transitioning to the \gls{MMR} shows large variability in 
the mass of enriched uranium and required \gls{SWU} capacity, because 
\glspl{MMR} require a large mass of enriched uranium when they are 
first deployed and do not require refueling. This behavior is not observed 
in the transitions to the Xe-100 reactor because this reactor utilizes 
on-line refueling, so fuel is continually sent to them. 

\section{Future Work}
One possible area of future work is extending the end date 
of the scenarios to 2125 to investigate how replacing deployed Xe-100 
reactors impacts the material requirements and energy produced in those 
transition scenarios. Another area of future work is to investigate the 
amount of \gls{HEU} required to produce the \gls{HALEU} required by 
each of the transitions. This will provide insight into how this \gls{HALEU}
production method can be used to meet the material requirements of these 
transition scenarios. 

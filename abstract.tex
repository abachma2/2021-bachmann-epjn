\begin{abstract}
The U.S. Department of Energy has outlined two ways to produce \gls{HALEU} 
for advanced
reactors: enriching natural uranium and downblending high-enriched uranium. 
To understand the material requirements of enriching natural uranium to 
produce \gls{HALEU}, this 
work simulates multiple fuel cycle scenarios to compare how the type of 
advanced reactor deployed and the energy growth demand affect the 
material requirements of the transition to \gls{HALEU}-fueled reactors. Fuel 
cycle scenarios considered include a no-growth and a 1\% growth transition to 
either the Ultra Safe Nuclear Corporation \gls{MMR} or the X-energy Xe-100 
reactor from the 
current fleet of Light Water Reactors. Materials of interest include the 
number of 
advanced reactors deployed, the mass of enriched uranium sent to the reactors, 
and the \gls{SWU} capacity required to enrich natural uranium for the reactors.
This work shows that deploying the \gls{MMR} requires a higher 
average mass and \gls{SWU} capacity than deploying Xe-100 reactors, but 
a lower average than the current Light Water Reactors. 
However, 
deploying the \gls{MMR} causes periods of large increases of required 
enriched uranium and \gls{SWU} capacity that are not present when 
deploying the Xe-100 reactor.
\end{abstract}

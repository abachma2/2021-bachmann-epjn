\begin{abstract}
The \gls{DOE} has outlined multiple ways to produce \gls{HALEU} for advanced
reactors that are anticipated to be commercially deployed. Selecting how 
to utilize these methods will depend on the advanced reactor(s) that are 
deployed and the transition scenario. To provide insight into this, this 
work simulates multiple fuel cycle scenarios to compare how the type of 
advanced reactor deployed and the transition scenario modeled affect the 
material requirements of the transition to \gls{HALEU}-fueled reactors. Fuel 
cycle scenarios considered include a no-growth and a 1\% growth transition to 
either the \gls{USNC} \gls{MMR} or the X-energy Xe-100 reactor from the 
current fleet of \glspl{LWR}. Materials of interest include the number of 
advaced reactors deployed, mass of enriched uranium sent to the reactors, 
and the \gls{SWU} capacity required to enrich natural uranium for the reactors.

The results of this work show that deploying the \gls{MMR} requires a higher 
average mass and \gls{SWU} capacity than deploying Xe-100 reactors, but 
a lower average than what is required by the current \glspl{LWR}. However, 
deploying the \gls{MMR} causes periods of large quantities of enriched 
uranium mass and required \gls{SWU} capacity that is not present when 
the Xe-100 reactor is deployed. 
\end{abstract}

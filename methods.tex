\section{Methodology}
This work simulates multiple transition scenarios to advanced reactors
requiring \gls{HALEU} for fuel, then quantifies and compares the front-end 
material requirements of each scenario. Five different fuel cycle scenarios 
are modeled using \Cyclus \cite{huff_fundamental_2016},
an open source, agent based fuel cycle simulator. The first scenario models 
only the \glspl{LWR} that are deployed in the United States, and provides 
a reference for comparison of the material requirements of the transitions. 
The next two scenarios model no-growth 
transitions to the \gls{USNC} \gls{MMR} or the X-Energy 
Xe-100 reactor. The last  
two scenarios model a 1\% annual growth transition to either the \gls{USNC} 
\gls{MMR} or the X-Energy Xe-100 reactor. Table \ref{tab:simulations} 
summarizes each of the scenarios.

\begin{table}[ht]
        \centering
        \caption{Summary of the fuel cycle scenarios}
        \label{tab:simulations}
        \begin{tabular}{c c c}
                \hline
                Simulation Number & Reactors Present & Growth \\\hline
                1 & \glspl{LWR} & N/A \\
                2 & \glspl{LWR} and \gls{USNC} \gls{MMR} & None \\
                3 & \glspl{LWR} and X-Energy Xe-100 reactor& None \\
                4 & \glspl{LWR} and \gls{USNC} \gls{MMR}& 1\% \\
                5 & \glspl{LWR} and X-Energy Xe-100 reactor& 1\% \\\hline

        \end{tabular}
\end{table}

The \gls{IAEA} \gls{PRIS} data base \cite{noauthor_power_1989} supplies
data about the currently deployed \glspl{LWR}. This database provides the 
grid connection date of each reactor deployed, power level, and the decommission 
date for any reactor closed before December 2020. Any reactor still in 
operation in December 2020 is assumed to operate for 60 years after it's 
grid connection date. Only reactors with a power level above 400 MWe are included 
to ensure that no research or experimental reactors are used. 
The mass of fuel to use in the \gls{LWR} reactor cores, including the mass of 
fuel required for each refueling, was obtained from \cite{todreas_nuclear_2012,cacuci_handbook_2010}.
All \glspl{LWR} are assumed to have an 18 month cycle length. 

We used a variety of open source documents to obtain data about the advanced reactors
\cite{harlan_x-energy_2018, hussain_advances_2018, mitchell_usnc_2020}. This includes 
the power output, enrichment level, fuel form, reactor lifetime, fuel 
burnup, and cycle time, shown in Table \ref{tab:reactor_summary}. 
The Xe-100 reactor has a much higher power output, requires a higher enrichment 
level, and has a longer reactor lifetime than the \gls{MMR}. However, the 
\gls{MMR} has a longer cycle time than the Xe-100 since it does not require 
refueling once it is operational. Both of the advanced reactors require fuel 
comprised of \gls{TRISO} particles, but in different forms. Refueling 
of the Xe-100 reactor is modeled as a replacement of 1/7th of the core mass 
every six months. 

\begin{table}[ht]
        \caption{Mico-reactor design specifications}
        \label{tab:reactor_summary}
        \begin{tabular}{l c c c}
            \hline
            Design Criteria & \gls{USNC} \gls{MMR} & 
                X-Energy Xe-100 \\\hline
            Reactor type & Modular HTGR & Modular HTGR \\
            Power Output (MWth) & 15 & 200 \\
            Enrichment (\% $^{235}U$) & 13 & 15.5 \\
            Cycle Length (years) & 20 & online refueling\\
            Fuel form & \gls{TRISO} compacts & \gls{TRISO} pebbles\\
            Reactor Lifetime & 20 years & 60 years \\
            Mass of uranium per refueling (kg) & 1912.9 & 223.87 \\
            Burnup (MWd/kg U) & 42.7 & 163 \\
            \hline
        \end{tabular}
    \end{table}

Each of the simulations model reactor deployment and operation from 1965 to 
2090, with the transition to advanced reactors beginning 
in 2025 for the applicable scenarios. Therefore, in the no growth scenarios 
the power demand is held constant 
at the power produced in 2025. The \glspl{LWR} are decommissioned as described 
above and the advanced reactors are deployed as needed to meet the prescribed 
power demand of the scenario. 

Fuel cycle facilities included in the simulations
range from the uranium mine to the final disposal of the spent fuel. Figure 
\ref{fig:fuel_cycle} shows the fuel cycle modeled in each simulation. Scenario 
1 only includes the facilities in blue. All facilities are used in Scenarios 
2-5 and the facilities in red are used only for material for the advanced reactors, 
such as \gls{HALEU} and \gls{TRISO} fuel. Although the back end of the fuel 
cycle is modeled, quantifying any waste amounts is considered outside the 
scope of this work. 

\begin{figure}[ht]
        \centering
        \begin{tikzpicture}[node distance=1.5cm]
            \node (mine) [facility] {Uranium Mine};
            \node (mill) [facility, below of=mine] {Uranium Mill};
            \node (conversion) [facility, below of=mill] {Conversion};
            \node (enrichment) [facility, below of=conversion]{Enrichment};
            \node (fabrication) [facility, below of=enrichment]{Fuel Fabrication};
            \node (reactor) [facility, below of=fabrication, xshift=-2cm]{Reactor};
            \node (adv_reactor) [transition, below of=fabrication, xshift=2cm]{Advanced Reactor};
            \node (wetstorage) [facility, below of=reactor]{Wet Storage};
            \node (drystorage) [facility, below of=wetstorage]{Dry Storage};
            \node (sinkhlw) [facility, below of=drystorage, xshift=2cm]{HLW Sink};
            \node (sinkllw) [facility, right of=enrichment, xshift=2cm] {LLW Sink};
    
            \draw [arrow] (mine) -- node[anchor=east]{Natural U} (mill); 
            \draw [arrow] (mill) -- node[anchor=east]{U$_3$O$_8$}(conversion); 
            \draw [arrow] (conversion) -- node[anchor=east]{UF$_6$}(enrichment);
            \draw [arrow] (enrichment) -- node[anchor=east]{Enriched U}(fabrication);
            \draw [arrow] (enrichment) -- node[anchor=south]{Tails}(sinkllw);
            \draw [arrow] (fabrication) -- node[anchor=east]{Fresh UOX}(reactor);
            \draw [arrow] (fabrication) -- node[anchor=west]{TRSIO fuel}(adv_reactor);
            \draw [arrow] (reactor) -- node[anchor=east]{Spent UOX}(wetstorage);
            \draw [arrow] (wetstorage) -- node[anchor=east]{Cool Spent UOX}(drystorage);
            \draw [arrow] (drystorage) -- node[anchor=east]{Casked Spent UOX}(sinkhlw);
            \draw [arrow] (adv_reactor) -- node[anchor=west]{Spent TRISO Fuel}(sinkhlw);
    
            \end{tikzpicture}
        \caption{Fuel cycle facilities and material flow between facilities. Facilities in 
        red are deployed in the transition scenarios.}
        \label{fig:fuel_cycle}
    \end{figure}

Recipes define the composition of each material going into and out of  
a facility. Spent \gls{LWR} fuel recipes assumes a 
51 MWd/MTHM burnup. Spent and fresh \gls{LWR} fuel recipes were obtained 
from \cite{yacout_visionverifiable_2006}. All other recipes capture the 
necessary isotopic ratios of uranium, but do not include other elements. 